\documentclass{beamer}
\usetheme{Berlin}
\usepackage{stmaryrd}

\makeatletter
\newsavebox{\@brx}
\newcommand{\llangle}[1][]{\savebox{\@brx}{\(\m@th{#1\langle}\)}%
  \mathopen{\copy\@brx\kern-0.5\wd\@brx\usebox{\@brx}}}
\newcommand{\rrangle}[1][]{\savebox{\@brx}{\(\m@th{#1\rangle}\)}%
  \mathclose{\copy\@brx\kern-0.5\wd\@brx\usebox{\@brx}}}
\makeatother

\expandafter\def\expandafter\insertshorttitle\expandafter{%
  \insertshorttitle\hfill%
  \insertframenumber\,/\,\inserttotalframenumber}

\title{A Powerful Active Attack on Supersingular Isogeny Diffie-Hellman (SIDH) Key Exchange}
\author{Ankit Kumar Misra \and Aman Singh}
\institute{CS 741: Advanced Network Security and Cryptography, \\Autumn 2022, IIT Bombay}
\date{\today}

\begin{document}

\frame{\titlepage}

%%%% How SIDH works

\begin{frame}{Introduction}
\begin{itemize}
    \item Supersingular Isogeny Key Exchange or SIKE is a proposed post-quantum cryptographic algorithm
    \item Smallest key sizes among its contenders
    \item Perfect forward secrecy
\end{itemize}
\end{frame}

\begin{frame}{Supersingular curves}
\begin{itemize}
    \item SIKE works in finite fields of the form $\mathbb{F}_{p^2} = \mathbb{F}_p(i)$ with elements represented as $u+vi$ where $u,v\in\mathbb{F}_p$
\end{itemize}
\begin{definition}
The curve $E_\lambda: y^2 = x(x-1)(x-\lambda)$ is supersingular if and only if $\lambda$ is a root of $H_p(x) = \sum_{i=0}^{(p-1)/2} {(p-1)/2 \choose i}^2 x^i$ such that $\lambda \neq 0,1$
\end{definition}
\begin{itemize}
    \item We are interested performing operations over j-invariants of supersingular elliptic curves in $\mathbb{F}_{p^2}$
\end{itemize}
\end{frame}

\begin{frame}{j-invariant}
\begin{definition}
    For a curve in Montgomery form $E_a: y^2 = x^3 + ax^2 + x$, 
    $$j(E_a) = 256\dfrac{(a^2-3)^3}{(a^2-4)}$$
\end{definition}
\begin{itemize}
    \item Isomorphic elliptic curves have equal j-invariant and curve with equal j-invariants are isomorphic
    \item Number of supersingular j-invariants in $\mathbb{F}_{p^2}$ is approximately $\lfloor p/12 \rfloor$ 
\end{itemize}
\end{frame}

\begin{frame}{Isogenies}
\begin{itemize}
    \item Isogenies are group homomorphisms on elliptic curves i.e they preserve the group structure. For an isogeny $\phi: E \rightarrow E'$
    $$\phi(A+B) = \phi(A) + \phi(B)$$
    \item Unlike isomorphisms, isogenies do not conserve the j-invariant
    \item The kernel $ker(\phi)$ of an isogeny contains the group elements that are mapped to the identity element $\mathcal{O}$ under $\phi$
    \item The order of an isogeny is the number of elements in its kernel 
\end{itemize}
\begin{definition}
    The d-torsion isogeny is the multiplication-by-d map represented as $[d]: E \rightarrow E'$ such that $[d]P = dP$. $E[d]$ is the kernel of $[d]$ in $E$.
\end{definition}
\end{frame}

\begin{frame}{Isogeny graph}
\begin{theorem}
$E[d] \cong \mathbb{Z}_d \times \mathbb{Z}_d$
\end{theorem}
\begin{theorem}
If $\phi: E \rightarrow E'$ is a separable isogeny of degree $d$, $ker(\phi) \subseteq E[d]$
\end{theorem}
\begin{itemize}
    \item To proceed further, we would like to compute 2-isogeny and 3-isogeny graphs starting from a fixed curve $E$
    \item A d-isogeny graph has vertices representing an equivalence class of elliptic curves with same j-invariant and an edge if there is an isogeny that connects them.
    \item Due to the above theorems, $\langle P, Q \rangle = E[2]$ and there exist three 2-isogenies from each $E$ with kernels $\langle P \rangle, \langle Q \rangle, \langle P+Q \rangle$
\end{itemize}
\end{frame}

\begin{frame}{Composing isogenies}
\begin{itemize}
    \item To implement SIDH we would need to compute isogenies of degree $2^{d}$ where $d \approx 200$
    \item This is done by composing $d$ 2-isogenies
    \item We wish to compute $\phi: E \rightarrow E/\langle S \rangle$ where $S$ is of order $2^d$
    \item As the first step, we calculate $S' = [2^{d-1}]S$ which has order 2. Associated with $S'$ is the isogeny $\phi': E \rightarrow E/\langle S' \rangle$ of degree 2 (i.e. a step in the 2-isogeny graph)
    \item Now, $0 = \phi'(S') = \phi'(2^{d-1}S) = 2^{d-1}\phi'(S)$. So, $\phi'(S)$ is a point of order $d-1$ in $E/\langle S' \rangle$
    \item By continuing this process, we can get an isogeny of degree $2^d$ in $d$ steps.
\end{itemize}
\end{frame}

\begin{frame}{SIDH protocol}
\begin{itemize}
    \item Prime $p$ is chosen to be of the form $2^{e_A}3^{e_B}-1$ with $2^{e_A} \approx 3^{e_B}$
    \item \textbf{Protocol parameters}: $E, P_A, Q_A, P_B, Q_B$ where $\langle P_A, Q_A \rangle = E[2^{e_A}]$ and $\langle P_B, Q_B \rangle = E[3^{e_B}]$
    \item Alice and Bob choose secret integers $k_A \in [0, 2^{e_A}-1], k_B \in [0, 3^{e_B}-1]$ and compute the isogeny $\phi_A$ and $\phi_B$ such that $ker(\phi_A) = \langle P_A + [k_A]Q_A \rangle, ker(\phi_B) = \langle P_B + [k_B]Q_B \rangle$ 
    \item \textbf{Public keys}: $PK_A = (\phi_A(E)=E_A, \phi_A(P_B), \phi_A(Q_B))$ and $PK_B = (\phi_B(E)=E_B, \phi_B(P_A), \phi_B(Q_A))$
    \item \textbf{Shared secret}: Alice computes $S_{BA} = \phi_B(P_A + [k_A]Q_A) = \phi_B(P_A) + [k_A]\phi_B(Q_A)$ and hence also $E_{BA} = \phi_B(E) / \langle S_{BA} \rangle$. Similarly, Bob calculates $E_{AB}$
\end{itemize}
\end{frame}

%%%% How the attack works

\begin{frame}{How can we attack SIDH?}
    \begin{itemize}
        \item Galbraith et al., 2016.
        \item If a party (say, Alice) does not change her private key $k_A$, her full private key can be recovered in $\mathcal{O}(\log p)$ interactions.
        \item For the attack, interactions with Alice are modeled in terms of accessing an oracle $O$:
        \begin{itemize}
            \item Input: $E_B, \phi_B(P_A), \phi_B(Q_A), E_{AB}$.
            \item Output: 1 if the $j$-invariant of $E_{AB}$ equals that of $E_B/\langle \phi_B(P_A)+[k_A]\phi_B(Q_A)\rangle$, and 0 otherwise.
        \end{itemize}
        \item Validation checks in SIDH can prevent basic attacks.
        \begin{itemize}
            \item Check that public key points $\phi_B(P_A), \phi_B(Q_A)$ have full order of $2^n$, to prevent small subgroup attacks.
            \item Weil pairing check for independence: $e_{2^n}(\phi_B(P_A),\phi_B(Q_A)) = e_{2^n}(P_A,Q_A)^{3^m}$.
        \end{itemize}
    \end{itemize}
\end{frame}

\begin{frame}{Extracting the LSB of $k_A$}
    \begin{itemize}
        \item For simplicity, let $R = \phi_B(P_A)$ and $S = \phi_B(Q_A)$.
        \item Attacker queries the oracle on $(E_B, R, S+[2^{n-1}]R, E_{AB})$.
        \item Oracle returns 1 if and only if $E_{AB}$ and $E_B/\langle R+k_A(S+[2^{n-1}]R)\rangle$ are isomorphic, i.e., $\langle R+k_A(S+[2^{n-1}]R)\rangle = \langle R+k_A S\rangle$.
    \end{itemize}
    \begin{lemma}
        Let $R,S\in E[2^n]$ be linearly independent points of order $2^n$, and $k_A\in\mathbb{Z}_{2^n}$. Then, $\langle R+k_A(S+[2^{n-1}]R)\rangle = \langle R+k_A S\rangle$ if and only if $k_A$ is even.
    \end{lemma}
\end{frame}

\begin{frame}{Extracting the LSB of $k_A$}
    \begin{lemma}
        Let $R,S\in E[2^n]$ be linearly independent points of order $2^n$, and $k_A\in\mathbb{Z}_{2^n}$. Then, $\langle R+[k_A](S+[2^{n-1}]R)\rangle = \langle R+[k_A] S\rangle$ if and only if $k_A$ is even.
    \end{lemma}
    \begin{proof}
        $(\implies)$ Groups generated by $R+[k_A](S+[2^{n-1}]R)$ and $R+[k_A] S$ are equal, so there exists $\lambda\in\mathbb{Z}^*_{2^n}$ such that
        \[ \lambda(R+[k_A](S+[2^{n-1}]R)) = R+[k_A] S. \]
        By linear independence of $R$ and $S$, we have $\lambda=1$, and thus $[k_A][2^{n-1}]R=0$. Since $R$ has order $2^n$, $k_A$ must be even.
        
        \medskip
        
        $(\impliedby)$ If $k_A$ is even, then $[k_A][2^{n-1}]R=0$. Hence proved.
    \end{proof}
\end{frame}

\begin{frame}{Extracting the LSB of $k_A$}
    \begin{itemize}
        \item The lemma just described implies that, for query $(E_B,R,S+[2^{n-1}]R,E_{AB})$, the oracle returns
        \begin{itemize}
            \item 1 if and only if $k_A$ is even.
            \item 0 if and only if $k_A$ is odd.
        \end{itemize}
        \item LSB of $k_A$ has been determined by a single oracle access.
        \begin{itemize}
            \item LSB = $1-$oracle's response
        \end{itemize}
        \item A similar strategy is adopted for the higher-order bits.
    \end{itemize} 
\end{frame}

\begin{frame}{Strategy to extract an arbitrary bit of $k_A$}
    \begin{itemize}
        \item Let $k_A = k_0 + 2^1k_1 + \dots + 2^{n-1}k_{n-1} = \mathcal{K}_i + 2^ik_i + 2^{i+1}k'$, where the $i$ LSBs (given by $\mathcal{K}_i$) are already known.
        \item The attacker queries $(E_B, [a]R+[b]S, [c]R+[d]S, E_{AB})$ for appropriately chosen $a,b,c,d$.
        \item Conditions to be satisfied:
        \begin{itemize}
            \item The oracle's response should reveal $k_i$.
            \item Attack should pass undetected through order checking and Weil pairing validation checks.
        \end{itemize}
    \end{itemize}
\end{frame}

\begin{frame}{Designing the attacker's query}
    \begin{itemize}
        \item More formally, the following have to be satisfied:
        \begin{itemize}
            \item If $k_i=0$ then $\langle [a+k_A c]R + [b+k_A d]S\rangle = \langle R + k_A S\rangle$.
            \item If $k_i=1$ then $\langle [a+k_A c]R + [b+k_A d]S\rangle \neq \langle R + k_A S\rangle$.
            \item $[a]R+[b]S$ and $[c]R+[d]S$ both have order $2^n$.
            \item $e_{2^n}([a]R+[b]S, [c]R+[d]S) = e_{2^n}(R,S) = e_{2^n}(P_A,Q_A)^{3^m}$.
        \end{itemize}
        \item The first three are satisfied by
        \begin{align*}
            a = 1 && b = -2^{n-i-1}\mathcal{K}_i\\
            c = 0 && d = 1 + 2^{n-i-1}
        \end{align*}
        \item Thus, $R' = [\theta](R - [2^{n-i-1}\mathcal{K}_i]S)$ and $S' = [\theta][1+2^{n-i-1}]S$.        
    \end{itemize}
\end{frame}

\begin{frame}{Designing the attacker's query}
    \begin{itemize}
        \item For the last condition, we need $e_{2^n}(R',S') = e_{2^n}(R,S)^{\theta^2(1+2^{n-i-1})} = e_{2^n}(R,S)$.
        \item So we use $\theta = \sqrt{(1+2^{n-i-1})^{-1}}\pmod{2^n}$.
        \item This square root can be shown to exist for $n-i-1\geq 3$, i.e., $i\leq n-4$.
        \item Oracle queries are used for $i = 0,1,\dots,n-4$, followed by brute force search over all 8 possibilities for $i=n-3,n-2,n-1$, checking whether $E/\langle R+k_A S\rangle$ equals $E_A$ in each case.
    \end{itemize}
\end{frame}

\begin{frame}{Implementation}
    \begin{itemize}
        \item We implemented SIDH and the attack in Sage, and verified their working for small values of $e_A$ and $e_B$.
        \item Code available at \url{https://github.com/ankitkmisra/SIDH}.
    \end{itemize}
\end{frame}

\begin{frame}{References}
    \begin{enumerate}
        \item \href{https://eprint.iacr.org/2016/859.pdf}{Galbraith et al., \textit{On the Security of Supersingular Isogeny Cryptosystems}, Asiacrypt 2016.}
        \item \href{https://2017.pqcrypto.org/school/slides/Isogeny_based_crypto.pdf}{PQCRYPTO's \textit{Summer School on Post-Quantum Cryptography} in Eindhoven, 2017.}
        \item \href{https://eprint.iacr.org/2019/1321.pdf}{Craig Costello, \textit{Supersingular Isogeny Key Exchange for Beginners}, Selected Areas in Cryptography (SAC) 2019.}
        \item \href{https://solvable.group/posts/isobeg/}{William Borgeaud's blog post on \textit{Isogeny-Based Crypto Part 1: The SIDH Protocol}, 2020.}
        \item \href{https://doc.sagemath.org/html/en/}{Sage documentation.}
    \end{enumerate}
\end{frame}



\end{document}